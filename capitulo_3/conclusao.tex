\section{Conclusão do Capítulo}\label{sec:conclusao_cap_3}

A primeira seção deste capítulo apresentou um breve resumo sobre o simulador the ONE, incluindo explanações sobre os protocolos implementados, arquitetura simplificada e funcionamento básico. Tal simulador é muito importante para o desenvolvimento da técnica objetivada, visto as facilidades oferecidas e o seu desempenho.

A linguagem Java, apresentada de forma sucinta na seção \ref{sec:java}, é de grande valia para a implementação da técnica, visto as abstrações oferecidas e a linguagem utilizada para o desenvolvimento do simulador.

A análise dos resultados da técnica é facilitada por meio dos relatórios gerados pelo simulador The ONE e da ferramenta gnuplot, apresentados nas seções \ref{sec:theONE} e \ref{sec:gnuplot}, respectivamente.

As ferramentas utilizadas são de grande importância para o projeto, contribuindo de forma significativa na construção da base necessária para o desenvolvimento e análise da técnica. Além disso, o estudo das ferramentas proporciona melhor entendimento e interação com assuntos relacionados às áreas de \emph{Simulação de Ambientes} e \emph{Probabilidade e Estatística}, permitindo adquirir conhecimentos não vistos durante o curso de Ciência da Computação.