\section{Conclusão do Capítulo}

Neste capítulo foram apresentados os cenários de teste considerados além dos resultados dos testes realizados. Neste trabalho, optou-se por definir os cenários de teste com base no utilizado por \cite{denis_artigo}, pois serviu como uma importante referência dentro literatura de DTNs. Além disso, foram considerados os dois estados principais da técnica, desligada e ligada, com o intuito de avaliar o seu comportamento quando comparada a uma rede que não a utiliza.

Acerca dos resultados, foi apresentado neste capítulo que o desempenho da técnica não é satisfatório quando analisa-se unicamente a economia das baterias. Entretanto, quando a recarga das baterias é considerada, a técnica é capaz de auxiliar no aumento da quantidade de mensagens entregues dependendo do intervalo de busca mínimo, padrão e máximo definidos. Neste trabalho, optou-se por testar apenas dois períodos, 
entretanto a busca por um período ideal é totalmente passível de ser realizada em trabalhos futuros.

Além disso, o aumento do consumo de energia relacionado ao aumento da entrega de mensagens por parte da técnica evidenciou que tornar inteligente o consumo das baterias não, necessariamente, significa economizá-las, mas sim consumí-la de acordo com a situação em que o nó se encontra. Ou seja, gastar em momentos oportunos e poupá-la em momentos não favoráveis.
