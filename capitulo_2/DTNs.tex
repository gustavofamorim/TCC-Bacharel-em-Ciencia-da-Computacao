\section{Redes Tolerantes a Atrasos e Desconexões}\label{sec:DTN}

Nesta seção é realizada uma breve análise sobre a arquitetura atual da Internet, seus protocolos e suas respectivas dificuldades para trabalhar em cenários para os quais não foram projetados. Além disso, é apresentada a fundamentação teórica necessária sobre as Redes Tolerantes a Atrasos e Desconexões, incluindo definição, arquitetura e os principais desafios existentes.

\subsection{Contexto}\label{sec:DTN_contexto}

A Internet é uma grande rede de computadores de imenso sucesso. A importância da sua organização é de fácil percepção, visto a grande quantidade de possibilidades oferecidas e o seu desempenho satisfatório. 

A arquitetura da Internet é baseada em um modelo referência denominado modelo OSI (do inglês "\emph{Open Systems Interconnection}" ou "Interconexão de Sistemas Abertos", em português), que estabelece uma arquitetura em camadas, sendo, cada uma delas, responsável por implementar serviços inerentes a uma determinada etapa da comunicação entre os dispositivos. 

Protocolos de comunicação são responsáveis por implementar os serviços de uma determinada camada. Tais protocolos são definidos por meio de RFCs (do inglês "\emph{Request For Comments}" ou "Pedido para Comentários", em português), documentos técnicos desenvolvidos e mantidos pela  IETF (do inglês "\emph{Internet Enginnering Task Force}" ou "Força Tarefa de Engenharia da Internet", em português), que, por sua vez, é uma instituição responsável por especificar os padrões a serem implementados na Internet.

Destacam-se, dentre todos os protocolos utilizados na Internet, o TCP (do inglês "\emph{Transmission Control Protocol}", ou "Protocolo de controle de transmissão") e o IP (do inglês "\emph{Internet Protocol}" ou "Protocolo da Internet"). Tais protocolos são a base da maioria das trocas de dados entre os hospedeiros\footnote{Hospedeiros são dispositivos que executam aplicações. Smartphones, laptops, computadores de mesa e PDAs são exemplos clássicos de hospedeiros\cite{kurose2010redes}.} da Internet. 

Sobre a Internet e seus dois principais protocolos, o TCP e o IP, valem as afirmações \cite{kurose2010redes}:

\begin{itemize}
    \item \textbf{Infraestrutura Dedicada:} Existem grandes corporações da área de telecomunicações e instituições sem fins lucrativos envolvidas na organização da arquitetura da Internet, seja realizando a sua manutenção ou criando novos protocolos, aumentando e melhorando a infraestrutura existente ou criando novos acordos internacionais visando a melhoria da transmissão de dados;
    \item \textbf{Existência de um caminho fim-a-fim:} Os protocolos da Internet seguem um modelo de comunicação que pressupõe a existência de ao menos um caminho fim-a-fim entre dois nós\footnote{Nós são os dispositivos que compõem uma rede, sendo os hospedeiros um exemplo deles.}. Tal caminho torna possível estabelecer uma conexão entre dois hospedeiros quaisquer conectados a Internet a qualquer momento. De fato, esse pressuposto é sempre atendido na Internet, principalmente pela imensa infraestrutura existente e os grandes investimentos realizados pelas corporações;
    \item \textbf{Confiabilidade:} O protocolo TCP tem como principal premissa a transferência confiável de dados entre os dispositivos. A confiabilidade é implementada utilizando técnicas de verificação de erros e retransmissão de mensagens quando algum erro é detectado;
    \item \textbf{Estabilidade:} Mesmo com a inexistência de garantias de atraso e largura de banda\footnote{Largura de banda é uma grandeza utilizada para medir o quão rápido se dá uma troca de dados entre dispositivos de computação. É mensurada como bits por segundo.}, a Internet é caracterizada por baixas taxas de atraso fim-a-fim e perdas de dados. Isso se deve ao fato de as empresas envolvidas sempre se preocuparem com a melhoria, manutenção e expansão da infraestrutura existente.
\end{itemize}

Existem cenários que as afirmações válidas na Internet não podem ser realizadas, como os ambientes definidos pela ausência de uma infraestrutura dedicada, problemas associados a mobilidade e limitações de hardware dos dispositivos. Esses ambientes comprometem seriamente o funcionamento dos protocolos TCP e IP, trazendo muitos desafios de conectividade entre os nós. Redes assim caracterizadas, como já introduzido, são conhecidas como DTNs (do inglês "Delay Tolerant Networks") ou \emph{Redes Tolerantes a Atrasos e Desconexões} \cite{alves2009redes}.

As DTNs são amplamente utilizadas para prover conexões em situações desafiadoras como as anteriormente apresentadas, possibilitando a troca de dados em lugares onde não é possível evitar altas taxas de latência e desconexões. Exemplos mais detalhados de aplicações são apresentados na subseção \ref{sec:aplicacoes}.

\subsection{Definições}\label{sec:definicoes}

As Redes Tolerantes a Atrasos e Desconexões são redes de computadores caracterizadas, principalmente, por conexões sem fio e ausência de infraestrutura dedicada para a troca de dados entre os dispositivos que a compõe, tornando necessário a cooperação dos nós para que ela seja autossuficiente. 

A mobilidade é uma importante característica das DTNs, tornando a topologia variante no tempo e, devido a falta de infraestrutura dedicada, existe a ausência de garantias de conexão intermitente entre os nós. As DTNs sofrem, ainda, com altas taxas de latência e desconexões \cite{fall2003delay}, sendo comum estarem na ordem de horas a, até mesmo, dias.

A utilização de dispositivos sem-fio limita a comunicação em um pequeno espaço geográfico, pois torna necessário que um nó esteja na área de alcance do outro para trocarem dados. Esse curto alcance dos dispositivos atrelado à mobilidade torna a cooperação um elemento básico nessas redes, pois, quando um remetente precisa contatar um outro nó que não está em sua área de alcance, este pode repassar a mensagem para um ou mais nós intermediários acessíveis, esperando que eles possam entregar a mensagem em um momento oportuno, seja para outros nós intermediários ou para o próprio destinatário. Confiar em outros nós para a entrega das mensagens aumenta as chances de sucesso na troca de dados, pois os nós intermediários podem mover-se para uma região que o remetente não costuma ir, mas o destinatário sim.

A Figura \ref{exemploComunicacaoSimplesDTN} apresenta um exemplo de comunicação em uma DTN, onde um remetente \emph{R} deseja enviar uma mensagem à um destinatário \emph{D}. Para isso, \emph{R} entra em contato com um nó intermediário \emph{a} e envia a mensagem para ele, confiando que este possa entregar a mensagem ao seu destino. O nó \emph{a}, por sua vez, repassa a mensagem para o nó \emph{b} que, ao entrar em contato com o destinatário, repassa a mensagem.

\begin{figure}[htp!]
\centering
    \begin{tikzpicture}
        [
            greennode/.style={circle, draw=green!60, fill=green!5, very thick, minimum size=15mm},
            yellownode/.style={circle, draw=yellow!60, fill=yellow!25, very thick, minimum size=15mm},
            bluenode/.style={circle, draw=blue!60, fill=blue!5, very thick, minimum size=15mm},
            squarednode/.style={rectangle, draw=red!60, fill=red!5, very thick, minimum size=5mm},
        ]
        
        %Nós
        \node[greennode]        (remetente)                                 {R};
        \node[yellownode]       (meioum)            [right=of remetente]    {a};
        \node[yellownode]       (meiodois)          [right=of meioum]       {b};
        \node[bluenode]         (destinatario)      [right=of meiodois]     {D};
         
        %Linhas
        \draw[->] (remetente.east) -- (meioum.west);
        \draw[->] (meioum.east) -- (meiodois.west);
        \draw[->] (meiodois.east) -- (destinatario.west);
    \end{tikzpicture}
    \caption{Exemplo de transmissão de dados em uma DTN}
    \label{exemploComunicacaoSimplesDTN}
\end{figure}

É importante ressaltar que, ao confiar em \emph{a}, \emph{R} não necessariamente apaga a mensagem de seu \emph{buffer}, pois ele próprio pode se encontrar com \emph{D} no futuro e entregar pessoalmente a mensagem. Essa característica também é apresentada por todos os nós da rede e é por essa razão que as DTNs são conhecidas também como redes "armazena e envia" (do inglês "store-and-forward")\cite{de2007redes, umrKehr}, como ilustrado na Figura \ref{armazena_envia}. 

\begin{figure}[htp!]
\centering
\includegraphics[width=0.9\textwidth]{figuras/cap_2/secao_1/armazena_envia.png}
\caption{Encaminhamento nas DTNs. Baseado em \cite{umrKehr}.}
\label{armazena_envia}
\end{figure}

A existência de dispositivos com hardware limitado torna importante o gerenciamento inteligente das mensagens acumuladas nos nós, pois, devido a limites de armazenamento, pode ser necessário excluir algumas em prol do armazenamento de outras. A mensagem que será apagada deve ser escolhida de forma a amenizar ocorrências de situações difíceis de prever, como apagar uma mensagem instantes antes de se encontrar com o destinatário.

Observa-se a importância dos contatos nas DTNs, pois a troca de mensagens só é possível na ocorrência dos mesmos. Como visto em \cite{rfc4838}, os contatos podem ser classificados de acordo com sua previsibilidade:

\begin{itemize}
    \item \textbf{Previsíveis:} Como o próprio nome indica, podem ser previstos de alguma forma. As redes espaciais são bons representantes para esta categoria devido ao fato de ser possível determinar, precisamente, o correto momento do envio e recepção de mensagens visto o comportamento cíclico dos astros;
    \item \textbf{Imprevisíveis:} São totalmente ou parcialmente estocásticos, ou seja, não é possível determinar, precisamente, quando e onde irão ocorrer. Redes onde os nós não possuem um padrão de comportamento bem definido representam esta categoria, sendo um exemplo as redes Ad Hoc.
\end{itemize}


Apesar de serem comumente encontradas na literatura como Redes Tolerantes a Atrasos e Desconexões, existem outras terminologias, tais como: Redes com Conectividade Eventual, Redes Desconectadas e Redes com Conectividade Transiente. Recentemente, o termo DTN também pode ser encontrado como Disruption-Tolerant Networking e é oriundo de atribuições realizadas pela DARPA, a Agência de Projetos de Pesquisa Avançada de Defesa dos Estados Unidos, que faz altos investimentos em pesquisas relacionadas às DTNs, principalmente para uso militar \cite{krishnan2007spindle, sehl2013viability}.

\newpage
\subsection{Arquitetura}

A definição de uma arquitetura básica capaz de atender e amenizar os problemas enfrentados nas DTNs é muito importante. Tal trabalho foi realizado por \cite{fall2003delay} e é alvo de inúmeros estudos e complementações. Em seu artigo, Fall propõe uma arquitetura baseada no armazenamento e reenvio de mensagens e utilização de \emph{gateways}, responsáveis por interconectar regiões\footnote{Regiões são definidas como áreas onde há uma maior homogeneidade entre as características dos nós.} de uma DTN.

Como visto em \cite{rfc5050} e \cite{umrKehr}, a troca de mensagens entre os nós é implementada, nas DTNs, por meio de um modelo em camadas baseado no modelo OSI, assim como a Internet. O principal diferencial entre o modelo das DTNs e o da Internet é a existência da \emph{Camada de Empacotamento} (do inglês "Bundle Layer"), ilustrada na Figura \ref{camadas_dtn}. Esta camada é responsável pelo armazenamento temporário de mensagens, gerenciamento do \emph{buffer} de armazenamento utilizado pelos nós e abstração da troca de mensagens pelas aplicações que rodam sobre DTNs.

\begin{figure}[htp!]
\centering
\includegraphics[width=0.7\textwidth]{figuras/cap_2/secao_1/camadas_dtn.PNG}
\caption{Pilha de camadas de comunicação das DTNs (esquerda) comparada com a da Internet (direita). Baseado em \cite{umrKehr}.}
\label{camadas_dtn}
\end{figure}

Os \emph{gateways} são dispositivos que compõem as DTNs capazes de armazenar e retransmitir mensagens utilizando diferentes protocolos de roteamento e transporte. Como exemplo, um \emph{gateway} pode receber uma mensagem de um nó utilizando os protocolos TCP e IP e retransmiti-la utilizando outros, como o SCTP e IP \cite{fall2003delay}. Seu uso se faz necessário pelo fato de DTNs de diferentes regiões executarem diferentes protocolos de comunicação, permitindo assim a interoperabilidade de protocolos existentes em arquiteturas distintas.

O modelo proposto por Fall prevê ainda o uso de tuplas para a identificação de objetos ou grupos deles. Elas possuem duas componentes. A primeira é responsável por identificar o nome de uma região e é estruturada de forma hierárquica e única. A segunda é responsável por identificar entidades ou grupo delas. Os \emph{gateways} ficam então responsáveis por armazenar tabelas de roteamento baseadas nessas tuplas, permitindo um mapeamento global, seja ele dinâmico ou estático, das regiões e otimização do redirecionamento das mensagens.

A Figura \ref{dtn_fall} ilustra a arquitetura proposta por Fall, onde existem quatro regiões (A, B, C e D) bem distintas, presença de diferentes dispositivos, identificados por tuplas, e meios de encaminhamento. O cenário apresentado é bastante diversificado, indo desde o transporte de mensagens por meio de um ônibus, identificado pela tupla \{B, R4\}, até a comunicação com um satélite artificial, identificado pela tupla \{D, Satellite\}.

\begin{figure}[htp!]
\centering
\includegraphics[width=1\textwidth]{figuras/cap_2/secao_1/dtn_fall.png}
\caption{Arquitetura de DTN utilizando gateways para interconectar DTNs com diferentes protocolos de comunicação \cite{fall2003delay}.}
\label{dtn_fall}
\end{figure}

O papel dos \emph{gateways} é reforçado na Figura \ref{dtn_fall}. Os \emph{gateways} identificados na região B como \{B, R3\} e \{B, R5\} armazenam temporariamente as mensagens enquanto o ônibus \{B, R4\} as transporta de um para o outro.

\subsection{Encaminhamento de Mensagens}
\label{subsec:encaminhamento_mensagens}
Protocolos de Encaminhamento de mensagens são importantes em qualquer arquitetura de redes de computadores. A partir deles as mensagens são comutadas nó a nó na rede até que elas sejam entregues aos seus destinos. Nas DTNs isso não é diferente e diversos protocolos de encaminhamento foram desenvolvidos voltados para as características dessas redes.

Os protocolos de encaminhamento destinados às DTNs podem ser divididos em dois grupos principais: Probabilísticos e Não-Probabilísticos. Cada grupo possui características próprias, como taxa de entregas de mensagens, taxa de utilização de buffers e controle da quantidade de cópias de mensagens existentes na rede.

\subsubsection{Protocolos Probabilísticos}

Protocolos Probabilísticos fazem uso de dados estatísticos calculados por meio de informações obtidas da própria rede, como taxa de encontro entre cada um dos nós, quantidade de retransmissões das mensagens, taxas de sucesso e insucesso das transmissões e quantidade de cópias existentes de uma determinada mensagem na rede. Esses protocolos utilizam critérios para determinar, a partir dos dados, os melhores critérios de encaminhamento a serem utilizados.

Por utilizarem dados estatísticos, os Protocolos Probabilísticos tendem a tomar melhores decisões de acordo com o reconhecimento da rede, pois a maior quantidade de dados obtida com o tempo melhora os dados estatísticos e, consequentemente, melhora as decisões.

Entre os Protocolos Probabilísticos, destaca-se o protocolo Prophet:

\begin{itemize}
    \item \textbf{Prophet:} De acordo com \cite{lindgren2003probabilistic}, este protocolo utiliza o histórico de encontros entre os dispositivos para a tomada de decisão de roteamento. Numa rede que utiliza este protocolo, cada um dos nós armazena uma lista com todos os nós que já foram encontrados e uma \emph{Previsibilidade de Entrega} que é calculada de acordo com a quantidade de encontros entre os nós e decai quando eles não entram em contato por um determinado tempo. Esse protocolo utiliza a propriedade da transitividade, pois se um determinado nó \emph{R} possui mensagens para um nó \emph{D} que ele nunca se encontrou, mas um nó intermediário \emph{I} se encontra constantemente com \emph{D}, para \emph{R} o nó \emph{I} terá maior relevância na propagação da mensagem do que um outro nó \emph{N} que nunca se encontrou com \emph{D};
    %\item \textbf{MaxProp \cite{burgess2006maxprop}:} Ainda sendo estudado.
\end{itemize}

\newpage

\subsubsection{Protocolos Não-Probabilísticos}

Protocolos Não-Probabilísticos caracterizam-se pela ausência do uso de dados estatísticos na retransmissão das mensagens, destacando-se os seguintes:

\begin{itemize}
    \item \textbf{Epidemic Routing:} Segundo \cite{vahdat2000epidemic}, este protoloco parte do pressuposto de que quanto mais cópias de uma mesma mensagem existirem na rede, maiores serão as chances de entrega. Numa rede que executa esse protocolo, os nós retransmitem todas as mensagens armazenadas em \emph{buffer} quando entram em contato com outro, inundando a rede com diversas cópias. Problemas como o congestionamento da rede e o estouro de \emph{buffers} são constantes, fazendo necessário o uso de técnicas para a redução desses problemas, como o uso de indicadores (\emph{beacons}) de entrega da mensagem, uso de políticas para priorizar umas em relação à outras (como o tempo de criação) e uso de um número máximo de saltos que uma mensagem pode sofrer;
    \item \textbf{Spray-and-wait:} Como visto em \cite{spyropoulos2005spray}, este protocolo foi desenvolvido com o intuito de reduzir a quantidade de duplicatas de uma mensagem na rede e as taxas de estouro de \emph{buffers} nos dispositivos. Basicamente, existem duas fases, uma de pulverização e outra de espera. Na fase de pulverização os nós vão disseminando a mensagem pela rede até que um número limitado de nós esteja armazenando a mensagem. Na fase de espera, os nós só retransmitem a mensagem quando o nó destino é encontrado. A mobilidade dos nós deve ser consideravelmente alta, pois nós pouco móveis podem acabar fazendo com que o destinatário nunca seja alcançado;
    \item \textbf{Direct Contact:} Em \cite{spyropoulos2007spray}, é visto que neste protocolo as mensagens só são retransmitidas se o remetente se encontra com o destinatário. Os problemas de congestionamento e estouro \emph{buffers} existente no protocolo epidêmico são praticamente resolvidos, mas as chances de entrega das mensagens são muito reduzidas, pois remetente e destinatário podem nunca se encontrar.
\end{itemize}

\subsubsection{Comparação entre os Protocolos Probabilísticos e Não-Probabilísticos}

Intuitivamente, percebe-se uma grande vantagem dos Protocolos Probabilísticos em relação aos Não-Probabilísticos.

Os Probabilísticos, com o uso de estatísticas provenientes da própria rede gerenciam de forma inteligente as retransmissões de mensagens, possuem baixas taxas de estouro de buffer, menor quantidade de duplicatas de mensagens e maiores taxas de entrega em relação aos Não-Probabilísticos. Sua característica adaptativa o torna mais eficiente de acordo com o tempo, mas, em redes recém implementadas, as probabilidades de erro e perdas de mensagens são elevadas.

Os protocolos Não-Probabilísticos baseiam-se primordialmente na mobilidade dos nós e em técnicas simples de roteamento de mensagens. Essas características os tornam problemáticos em redes com nós mais estáticos, mas de implementação simplificada em relação aos Probabilísticos. Mesmo aumentando as chances de entregas, a inundação da rede realizada pelo Protocolo Epidêmico pode acabar afogando os nós, fazendo com que o desempenho não seja satisfatório devido às altas taxas de estouro dos buffers.

\subsection{Aplicações Existentes}
\label{sec:aplicacoes}

Na literatura sobre DTNs inúmeros casos de aplicações são referenciados. Nesta subseção são apresentados alguns desses casos.

\subsubsection{O ZebraNet}

O projeto ZebraNet tem como objetivo monitorar a vida de zebras. Por meio de colares colocados nesses animais é feita a coleta e o armazenamento de dados referentes às suas vidas em seu habitat, permitindo uma análise detalhada do comportamento deles \cite{juang2002energy} \cite{zhang2004hardware}.

Assim como todo animal em seu ambiente natural, as zebras se movimentam de uma forma que a coleta de dados por humanos é dificultada, principalmente pelo fato de que os seres humanos afugentam esses animais. A solução para isso é obtida por meio do uso de colares que são colocados nas zebras e funcionam como nós de uma grande rede DTN. 

O encaminhamento das mensagens pelos dispositivos é feito de forma hierárquica. A posição dos nós dentro dessa hierarquia é baseada na quantidade de vezes que este se encontrou com a estação de coleta de dados. Quanto maior essa quantidade de encontros com a base, mais próximo do topo o nó estará da hierarquia. Estar próximo do topo significa que o nó tende a se encontrar muito com a base e, consequentemente, possui grande chance de se encontrar novamente num futuro próximo. A partir daí, quando um nó se encontra com dois dispositivos, um próximo do topo e outro distante, será dada a preferência de encaminhamento para o nó mais próximo do topo, visto que maiores são as chances dele se encontrar com a base num futuro próximo \cite{juang2002energy}.

A Figura \ref{zebra} apresenta uma zebra com o colar desenvolvido pelo projeto. Pode ser visto que ele foi desenvolvido para se adequar ao pescoço desses animais. Além disso, instalar esses dispositivos nesses animais exige que ele seja o mais confortável possível, visto que qualquer irritação gerada pelo mesmo pode influenciar nos resultados obtidos pelo projeto.

\begin{figure}[htp!]
    \centering
    \includegraphics[width=0.8\textwidth]{figuras/cap_2/secao_1/zebranet.png}
    \caption{Zebra com o colar desenvolvido pelo projeto ZebraNet \cite{zhang2004hardware}.}
    \label{zebra}
\end{figure}

O estudo da vida das zebras por meio do ZebraNet trás contribuições para diversas áreas de estudo, como ecologia, biologia e redes de computadores.

\subsubsection{Redes Móveis Ad Hoc}

Um caso de aplicação das DTNs são as Redes Móveis Ad Hoc (do inglês "Mobile Ad Hoc Netwrok" ou MANET). Em geral, uma MANET, é uma infraestrutura autônoma composta por dispositivos móveis sem fio que trocam dados sem a necessidade de um elemento intermediador, como nas redes WiFi em modo Ad Hoc.

\subsubsection{O Wizzy Digital Courier}

Prover conectividade em áreas muito distantes de qualquer infraestrutura de Internet é um grande desafio e, já não é de hoje, que conectar regiões emergentes, como áreas rurais, atraem a atenção de pesquisadores \cite{tierProject}.

O Wizzy Digital Courier, nada mais é do que um projeto responsável por prover conectividade com a Internet por meio de uma DTN a escolas localizadas em locais remotos na África do Sul. Para isso, o projeto faz uso de uma motocicleta, equipada com dispositivos de armazenamento móveis, que realiza o encaminhamento dos dados até a cidade mais próxima que possui conexão com a Internet de alta capacidade. Geralmente, o processo de encaminhamento das mensagens com a motocicleta leva muitas horas, todavia é uma forma barata e acessível para os padrões econômicos da região. \cite{jain2004routing}


\subsubsection{Comunicação Espacial}

A comunicação entre satélites em órbita, estações, naves espaciais, sondas interplanetárias e telescópios depende da movimentação destes elementos e dos próprios movimentos dos corpos celestes, como planetas e estrelas. Os movimentos de rotação e translação dos planetas, estrelas e satélites naturais podem, eventualmente, gerar períodos de conexão e desconexão com centros localizados na Terra. A mobilidade e frequentes desconexões são características são típicas das DTNs e são marcantes neste cenário.

A Figura \ref{satelites} mostra um exemplo de sistema de comunicação interplanetária composto por dois satélites artificiais. Em um determinado instante \emph{a} do tempo, os satélites conseguem trocar dados pois não existem barreiras físicas suficientemente fortes para inviabilizar a comunicação. No instante \emph{b} a comunicação é inviabilizada pois o planeta marciano encontra-se entre os dois dispositivos.

\begin{figure}[htp!]
    \centering
    \includegraphics[width=1\textwidth]{figuras/cap_2/secao_1/satelites.png}
    \caption{Exemplo de comunicação espacial.}
    \label{satelites}
\end{figure}

\subsection{Principais Desafios}

As DTNs são conhecidas ainda como CHANTS ("Challenged NeTworks", ou "Redes com Desafios", em português) \cite{chen2006hybrid}. Tal atribuição remete, principalmente, a grande quantidade de desafios envolvidos na gerência e implementação de protocolos de roteamento para esse tipo de rede.

A característica móvel dos dispositivos torna necessário o uso de baterias que possuem capacidade de armazenamento de carga limitada. Essa limitação é um desafio inerente das DTNs e depende do uso de mecanismos de recarga e desenvolvimento de técnicas visando o consumo inteligente da energia por parte dos dispositivos.

O gerenciamento inteligente do consumo das baterias é, atualmente, alvo de grandes pesquisas, como pode ser visto em \cite{denis_artigo}. Silva, em seu artigo, realiza uma análise do consumo dos principais protocolos de roteamento das DTNs e, como principal resultado, afirma que a busca por dispositivos é grande responsável pelo consumo das baterias e aponta um intervalo ótimo de 32 segundos entre as buscas.

Em \cite{denis_artigo} não são considerados fatores como a probabilidade de contatos em regiões geográficas. A busca por dispositivos em áreas de maior quantidade de contatos registrada possui maior probabilidade de sucesso do que em áreas com menor quantidade, logo é de importante valia o estudo da dinamização do intervalo de busca proposto por Silva. Para tanto, podem ser utilizadas teorias relacionadas ao \emph{Referenciamento Geográfico} e \emph{Gradiente de Concentração} para o mapeamento da quantidade de contatos das regiões geográficas. Essas teorias são explanadas nas seções subsequentes.
