\section{Conclusão do Capítulo}\label{sec:conclusao_cap_2}

A primeira seção deste capítulo apresentou um breve estudo sobre as DTNs, incluindo conceitos relacionados, arquitetura, aplicações e uma análise sobre o problema de energia existente. Além disso, ainda sobre as DTNs, os principais protocolos de disseminação foram descritos de forma sucinta.

Conceitos relacionados aos Gradientes de Concentração foram apresentados na seção subsequente a das DTNs. A explanação intuitiva foi utilizada visando facilitar o entendimento da analogia utilizada para a aplicação no ambiente das redes objetivadas neste trabalho.

Foi visto que o estudo e dinamização do intervalo de busca proposto por \newline \cite{denis_artigo} pode ser realizado a partir da análise da incidência de contatos em regiões previamente estabelecidas. Para tanto, foram apresentados conceitos sobre referenciamento geográfico, descrevendo sucintamente como referenciar posições no globo e as principais tecnologias de referenciamento existentes atualmente.

A importância da análise incremental apresentada neste capítulo se dá pelo principal problema existente nas DTNs: o consumo de energia. Não existem, atualmente, formas de solucioná-lo definitivamente, mas é possível desenvolver técnicas de software que visam realiza o consumo inteligente das baterias, que é o principal objetivo do presente trabalho.